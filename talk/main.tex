\documentclass{beamer}
\usepackage{scipy}
\usepackage{bera}
\usepackage{listings}
\usepackage[table, x11names, rgb]{xcolor}
\definecolor{keywords}{RGB}{255,0,90}
\definecolor{comments}{RGB}{60,179,113}
\definecolor{background}{RGB}{250,250,250}
\lstset{language=Python,
        basicstyle={\ttfamily,\small},
        frame=tb,
        keywordstyle=\color{keywords},
        commentstyle=\color{comments},
        showstringspaces=false,
        backgroundcolor=\color{background}}
\usepackage{dot2texi}
\usepackage{tikz}
\usetikzlibrary{shapes,arrows}
\usepackage{hyperref}
\usepackage{amsmath, amssymb}

% --------------------------------------------
% Sudoku
% --------------------------------------------

\renewcommand*\sudokuformat[1]{\sffamily#1}
\setlength\sudokusize{5cm}


\title{\talktitle}
\author{\Us}
\date{\talkdate}

\begin{document}

\begin{frame}
  \titlepage
  \conferencebanner
\end{frame}

\begin{frame}
\frametitle{Background - The OKlibrary}
 \begin{itemize}
  \item<1-> \url{http://www.ok-sat-library.org/}
  \item<2-> Open-source research platform and generative library for generalised SAT solving
  \item<3-> Developed by Oliver Kullmann at the University of Swansea since 2005.
  \item<4-> Modeling of combinatorial puzzles via SAT (Latin squares/Sudoku)
  \item<5-> C++/Lisp/Bash
 \end{itemize}
\end{frame}

\begin{frame}
\frametitle{Background - \texttt{sudoku.py}}
 \begin{itemize}
  \item<1-> \url{http://bitbucket.org/matthew/scipy2010}
  \item<2-> Open-source 
  \item<3-> Developed by authors at Berea College since 2010
  \item<4-> Modeling of Sudoku puzzles in a variety of domains
  \item<5-> Python/IPython
 \end{itemize}
\end{frame}

\begin{frame}
\frametitle{Overview}
 \begin{itemize}
  \item<1-> {Modeling Sudoku in Python
    \begin{itemize}
      \item<2-> Constraint models
      \item<3-> IP models
      \item<4-> Graph models
      \item<5-> Polynomial models
    \end{itemize}}
  \item<6-> {Using \texttt{sudoku.py} 
    \begin{itemize}
      \item<7-> Creating puzzles
      \item<8-> Solving puzzles
      \item<9-> Writing scripts
    \end{itemize}}
 \end{itemize}
\end{frame}

\begin{frame}
\frametitle{Introduction - Sudoku}
  The latin property.
\end{frame}

\begin{frame}
 \begin{columns}[c] 
  \column{.5\textwidth} 
   A \emph{traditional Sudoku puzzle} is a partial assignment of $1,\dots,9$ to the cells of a $9\times 9$ grid with the latin property on rows, columns and boxes.
  \column{.5\textwidth}
   \begin{figure}[h]
    \centering
    \sudokuexampleone
    \caption{Sudoku puzzle}
   \end{figure} 
 \end{columns}
\end{frame}

\begin{frame}
 \begin{columns}[c]
  \column{.5\textwidth} 
   A \emph{(generalized) Sudoku puzzle} of boxsize $n$ is a partial assignment of $1,\dots,n^2$ to the cells of an $n^2\times n^2$ grid with the latin property on rows, columns and boxes.
  \column{.5\textwidth}
   \begin{figure}[h]
    \centering
    \sudokuexampleone
    \caption{Sudoku puzzle}
   \end{figure} 
 \end{columns}
\end{frame}

\begin{frame}
 \begin{columns}[c] 
  \column{.5\textwidth} 
   A \emph{solution} of a Sudoku puzzle is a total assignment which extends the original partial assignment and satisfies the same latin properties.
  \column{.5\textwidth}
   \begin{figure}[h]
    \centering
    \sudokuexampletwo
    \caption{Sudoku puzzle}
   \end{figure} 
 \end{columns}
\end{frame}

\begin{frame}[fragile]
\frametitle{Modeling Sudoku -- Basics}
 \begin{figure}[h]
  \centering
  \librarytable
  \caption{Table}
 \end{figure} 

 \begin{figure}[h]
  \centering
  \lstinputlisting[firstline=15,lastline=17,frame=tb]{../src/sudoku.py}
  \caption{Code example}
 \end{figure} 
\end{frame}

\begin{frame}
\frametitle{Modeling Sudoku -- Constraints}

\end{frame}

\begin{frame}[fragile]
\frametitle{Modeling Sudoku -- Graphs}
 \begin{figure}[h]
  \centering
  \begin{dot2tex}[circo,mathmode,options={--graphstyle "scale=0.25"}]
   \input{../dot/empty_sudoku_graph_2.dot}
  \end{dot2tex}
  \caption{The Shidoku graph}
 \end{figure}
\end{frame}

\begin{frame}
\frametitle{Modeling Sudoku -- Linear systems}

\end{frame}

\begin{frame}
\frametitle{Modeling Sudoku -- Polynomials}

\end{frame}

\begin{frame}
\frametitle{The future -- Additional models}
 \begin{itemize}
  \item <1-> Exact cover problems
  \item <2-> SAT models
 \end{itemize}
\end{frame}

\begin{frame}
\frametitle{The future -- Experimental investigations}
 \begin{itemize}
  \item <1-> Minimally uniquely completable puzzles
 \end{itemize}
\end{frame}

\begin{frame}
\frametitle{The future -- Algorithms}
 \begin{itemize}
  \item <1-> A Sudoku database
  \item <2-> Data-mining
  \item <3-> Hybrid/meta algorithms
 \end{itemize}
\end{frame}

\begin{frame}
  The End -- Thank You!
\end{frame}

\end{document} 

