In this section we introduce several models of Sudoku. The models introduced here are implemented in the open-source library (\texttt{sudoku.py}) developed by the authors. Demonstrations of the library components corresponding to each of the different models are given.

\begin{center}
 \librarytable
\end{center}

\subsection{Constraint models}
\label{sec:models:constraints}

Constraint models for Sudoku puzzles are discussed in \cite{simonissudoku}. The simplest model uses the \texttt{all\_different} constraint.

In Listing \ref{constraintdemo}, an example is shown of how to model a Sudoku puzzle with \texttt{sudoku.py} and use the built-in constraint solver of \texttt{python-constraint} to find a solution.

\lstinputlisting[firstline=45,lastline=59,caption=Solving a Sudoku via constraint model,label=constraintdemo]{../../demos/article_demo.py}

\subsection{Graph models}
\label{sec:models:graph}

A graph model for Sudoku is presented in \cite{gagovargaset}. In this model, every cell of the Sudoku grid is represented by a vertex. The edges of the graph are given by the cell dependency relations. In other words, if two cells lie in the same row, column or box, then their vertices are joined by an edge in the graph model.

\begin{figure}[h]
\centering
\begin{dot2tex}[circo,mathmode,options={--graphstyle "scale=0.40"}]
  \input{../../dot/empty_sudoku_graph_2.dot}
\end{dot2tex}
\caption{The Shidoku graph}
\end{figure}

\subsection{Polynomial system models}
\label{sec:models:polynomials}

The graph model in \cite{gagovargaset} can be used to model a Sudoku puzzle as a system of polynomial equations. The polynomial system model presented in \cite{gagovargaset} consists of a polynomial for every vertex in the graph model and a polynomial for every edge. The vertex polynomials have the form $F(x_j) = \prod_{i=1}^{9} (x_j - i)$. The edge polynomials are $G(x_i, x_j) = \frac{F(x_i) - F(x_j)}{x_i - x_j}$, where $x_i$ and $x_j$ are adjacent vertices in the graph model. 

\subsection{Integer/linear programming models}
\label{sec:models:lp}

In \cite{Bartlett2008} a model of Sudoku as an integer programming problem is presented. In this model, the variables are all binary.

\begin{equation}
  x_{ijk} \in \{0, 1\}
\end{equation}

Variable $x_{ijk}$ represents the assignment of symbol $k$ to cell $(i,j)$ in the Sudoku puzzle.

\begin{equation}
 x_{ijk} = 
  \left\lbrace 
   \begin{array}{rl}
    1 & \mbox{ if cell $(i, j)$ contains symbol $k$} \\
    0 & \mbox{ otherwise}
   \end{array}
  \right.
\end{equation}

This model has a set of equations which force every solution to assign a symbol to every cell in the finished Sudoku puzzle.

\begin{equation}
 \sum_{k = 1}^{n} x_{ijk} = 1, \quad 1 \leq i \leq n, 1 \leq j \leq n
\end{equation}

Fixed elements in the Sudoku puzzle, given by a set $F$ of triples $(i,j,k)$, are each represented by an equation in the system:

\begin{equation}
  x_{ijk} = 1, \quad \forall (i,j,k) \in F
\end{equation}

The remaining equations in this model represent the unique occurence of every symbol in every column:

\begin{equation}
 \sum_{i = 1}^{n} x_{ijk} = 1, \quad 1 \leq j \leq n, 1 \leq k \leq n
\end{equation}

every symbol in every row:

\begin{equation}
 \sum_{j = 1}^{n} x_{ijk} = 1, \quad 1 \leq i \leq n, 1 \leq k \leq n
\end{equation}

and every symbol in every box:

\begin{equation}
 \sum_{j = mq - m + q}^{mq} \sum_{i = mp - m + 1}^{mp} x_{ijk} = 1, \quad 1 \leq k \leq n, 1 \leq p \leq m, 1 \leq q \leq m
\end{equation}

\lstinputlisting[firstline=60, lastline=76, caption=Solving a Sudoku via LP model,label=constraintdemo]{../../demos/article_demo.py}


